\documentclass{jarticle}

\usepackage[dvipdfmx]{graphicx}
\usepackage{url}
\usepackage{listings,jlisting}
\usepackage{ascmac}
\usepackage{amsmath,amssymb}

%ここからソースコードの表示に関する設定
\lstset{
    basicstyle={\ttfamily},
    identifierstyle={\small},
    commentstyle={\smallitshape},
    keywordstyle={\small\bfseries},
    ndkeywordstyle={\small},
    stringstyle={\small\ttfamily},
    frame={tb},
    breaklines=true,
    columns=[l]{fullflexible},
    numbers=left,
    xrightmargin=0zw,
    xleftmargin=3zw,
    numberstyle={\scriptsize},
    stepnumber=1,
    numbersep=1zw,
    lineskip=-0.5ex
}
%ここまでソースコードの表示に関する設定 

\title{知能プログラミング演習II 課題2}
\author{グループ07\\
    29114031 大原 拓人\\
%  {\small (グループレポートの場合は、グループ名および全員の学生番号と氏名が必要)}
}
\date{2019年11月5日}

\begin{document}
\maketitle

\paragraph{提出物} 個人レポート、グループプログラム「group07.zip」
\paragraph{グループ} グループ07
\paragraph{メンバー}
\begin{tabular}{|c|c|c|}
    \hline
    学生番号&氏名&貢献度比率\\
    \hline\hline
    29114007&池口弘尚&0\\
    \hline
    29114031&大原拓人&0\\
    \hline
    29114048&北原太一&0\\
    \hline
    29114086&飛世裕貴&0\\
    \hline
    29114095&野竹浩二朗&0\\
    \hline
\end{tabular}

\section{課題の説明}
\begin{description}
    \item[必須課題3-1] セマンティックネットのプログラムを参考に,
    グループメンバー全員(およびその周辺人物)についてのセマンティックネットを構築せよ.
    個人レポートには自分のみ(とその周辺)に関するセマンティックネットを示し,
    グループレポートには全員(とその周辺)に関するセマンティックネットを示せ.
    \item[必須課題3-2] フレームのプログラムを参考に,自分達の興味分野に関する
    知識をフレームで表現せよ.その分野の知識を表す上で必須となるスロットが
    何かを考え,クラスフレームを設計すること.
    個人レポートには自分が作ったインスタンスフレームのみ(クラスフレームの
    設計担当者はクラスフレームも)を示し,グループレポートにはクラスフレーム
    および全員分のインスタンスフレームを示せ.
    \\ 注: クラスフレームの設計についてはグループ内で共通にし,
    「メンバー毎にスロット定義がバラバラ」等ということがないよう気をつけること.
    \item[必須課題3-3] 課題3-1または3-2で作った知識表現を用いた質問応答システムを作成せよ.
    \\ なお,ユーザの質問は英語や日本語のような自然言語が望ましいが,難しければ課題2で扱ったような変数を含むパターン (クエリー) でも構わない. 
\end{description}


\section{課題3-1}
\begin{screen}
    セマンティックネットのプログラムを参考に,
    グループメンバー全員(およびその周辺人物)についてのセマンティックネットを構築せよ.
    個人レポートには自分のみ(とその周辺)に関するセマンティックネットを示し,
    グループレポートには全員(とその周辺)に関するセマンティックネットを示せ.
\end{screen}
\subsection{手法}
    自身に関係するセマンティックネットを構築した。構築したセマンティックネットは
    パワーポイントで視覚的に表現した。
\subsection{実装}
    自分に関するセマンティックネットは以下のようになった。
\begin{center}
    \includegraphics[scale=0.3]{kadai3SN.pdf}
\end{center}
\subsection{考察}
    出身地と、所有する携帯の種類は上位クラスが共通しそうな
    項目として加えた。ほかの項目は自分の趣向に合わせて作ったので、
    グループのメンバーと共通するものはほとんどなかった。

\section{課題3-2}
\begin{screen}
    フレームのプログラムを参考に,自分達の興味分野に関する
    知識をフレームで表現せよ.その分野の知識を表す上で必須となるスロットが
    何かを考え,クラスフレームを設計すること.
    個人レポートには自分が作ったインスタンスフレームのみ(クラスフレームの
    設計担当者はクラスフレームも)を示し,グループレポートにはクラスフレーム
    および全員分のインスタンスフレームを示せ.
    \\ 注: クラスフレームの設計についてはグループ内で共通にし,
    「メンバー毎にスロット定義がバラバラ」等ということがないよう気をつけること.
\end{screen}
\subsection{手法}
    グループで共通させる項目を決めた上で、いくつかの
    ゲームに出現するキャラクターについてクラスフレームを作成した。
\subsection{実装}
    グループで共通させる項目を決めた上で、私はポケモンのキャラクターに
    関する以下のようなクラスフレームを作成した。
    \begin{center}
        \includegraphics[scale=0.3]{kadai3pokemon.pdf}
    \end{center}
\subsection{考察}
    話題をゲームのキャラクターにすることで、タイトル、性別、武器の有無、
    登場するゲームタイトルという、キャラクターが思いつけばすぐに
    スロットの値がわかる項目を設定することができた。

\section{課題3-3}
\begin{screen}
    課題3-1または3-2で作った知識表現を用いた質問応答システムを作成せよ.
    \\ なお,ユーザの質問は英語や日本語のような自然言語が望ましいが,難しければ課題2で扱ったような変数を含むパターン (クエリー) でも構わない. 
\end{screen}
\subsection{手法}
セマンティックネットを構築できる簡単な日本語のみを受け付けて、
それをラベルとリンク元、リンク先に分解して、linkクラスのadd
メソッドに引数として与えられるようにする。

\subsection{実装}
私の担当した部分は、時間内に実行できるコードを作成できなかったので考察のみ示す。
\subsection{考察}
セマンティックネットとして表現し直せる文章はExample.javaの例を用いて
以下の3種類に分けられる。
\begin{description}
    \item[主語+もの+動詞] 「太郎はフェラーリを所有している」
    \item[主語+もの+である] 「太郎は名工大の学生である」
    \item[主語(もの+連体助詞+属性)+もの] 「太郎の趣味は野球である」  
\end{description}
先の2つは文節ごとの名詞のかたまりを抽出してLinkクラスのaddメソッドに
動詞(またはis-a)、主語、ものの順番で引数として渡せばよい。
しかし、3つ目は引数として渡す順番が変わってしまうので
この構文であることを検出しなければならない。また、連体詞「の」を
用いて検出する場合、「名工生の学生は学生である」という文章が
正しく解釈できない。

\section{感想}
今回はグループで集まって課題をするための時間を多く作れなかったので、
進捗が遅れてしまった。日本語の構文解析ができるmecabやcabochaを用いたが、
得られる情報が多く混乱してしまった。自分に必要な情報を整理できるようにしたい。

% 参考文献
\begin{thebibliography}{99}
    \bibitem{pl} ウェブインテリジェンスの演習で用いられたコードの例を参考にした
\end{thebibliography}

\end{document}